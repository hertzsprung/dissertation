\section{Advection}
Following \textcite{schaer2002}, a scalar anomaly is transported above the orography by solving the advection equation for a prescribed horizontal wind.  The domain is \SI{300}{\kilo\meter} wide and \SI{25}{\kilo\meter} high.  The terrain is wave-shaped, specified by the surface height $\surface$ such that
\begin{align}
	\surface(x) &= \cos^2 \left( \frac{\pi x}{\lambda} \right) \surface^\star
%
	\intertext{where}
%
	\surface^\star(x) &= \left\{ \begin{array}{l l}
		h_0 \cos^2 \left( \frac{\pi x}{2a} \right) & \quad \text{if $| x | < a$} \\
		0 & \quad \text{otherwise}
	\end{array} \right.
\end{align}
where $a = \SI{25}{\kilo\meter}$ is the mountain half-width, $h_0 = \SI{3}{\kilo\meter}$ is the maximum mountain height, and $\lambda = \SI{8}{\kilo\meter}$ is the wavelength.  On the SLEVE grid, the large-scale component $\surface_1$, as described in section~\ref{sec:theory:sleve}, is given by
\begin{align}
	\surface_1(x) &= \frac{1}{2}\surface^\star(x)
\end{align}
\TODO{describe wind field, then tracer}


\TODO{
\begin{itemize}
\item Advection of tracer blob above orography (advection equation, not real dynamics)
\item No wind near ground, uniform zonal wind aloft
\item Compared meshes, and vanLeer versus tvdLimitedCubicUpwindCPCFit.  Compared against noOrography as a baseline.
\item Looked at l2 error norm at t=10000s by comparing with analytic tracer (created with setScalarOverOrography)
\item Cut-cell style meshes, unsurprisingly, same l2 error as noOrography
\item tvdLimitedCubicUpwindCPCFit slightly better than vanLeer, but much more computationally expensive
\end{itemize}
}
