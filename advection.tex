\section{Advection}
\TODOsidenote{check SLEVE grid isn't optimised -- exponent in add2dMountainDict should equal 1}
Following \textcite{schaer2002}, a scalar anomaly is transported above the orography by solving the advection equation for a prescribed horizontal wind.
The wind profile, terrain profile and initial scalar anomaly are shown in Figure~\ref{fig:advection:initial}.

\subsection{Specification}
The domain is \SI{300}{\kilo\meter} wide and \SI{25}{\kilo\meter} high.  The terrain is wave-shaped, specified by the surface height $\surface$ such that
\begin{align}
	\surface(x) &= \cos^2 \left( \frac{\pi x}{\lambda} \right) \surface^\star
%
	\intertext{where}
%
	\surface^\star(x) &= \left\{ \begin{array}{l l}
		h_0 \cos^2 \left( \frac{\pi x}{2a} \right) & \quad \text{if $| x | < a$} \\
		0 & \quad \text{otherwise}
	\end{array} \right.
\end{align}
where $a = \SI{25}{\kilo\meter}$ is the mountain half-width, $h_0 = \SI{3}{\kilo\meter}$ is the maximum mountain height, and $\lambda = \SI{8}{\kilo\meter}$ is the wavelength.  On the SLEVE grid, the large-scale component $\surface_1$, as described in section~\ref{sec:theory:sleve}, is given by
\begin{align}
	\surface_1(x) &= \frac{1}{2}\surface^\star(x)
\end{align}
For comparison, the same tests were performed with no orography, such that $\surface = \SI{0}{\kilo\meter}$ everywhere.

The wind is entirely horizontal and is prescribed as
\begin{align}
	u(z) = u_0 \left\{ \begin{array}{l l}
		1 & \quad \text{if $z \geq z_2$} \\
		\sin^2 \left( \frac{\pi}{2} \frac{z - z_1}{z_2 - z_1} \right) & \quad \text{if $z_1 < z < z_2$} \\
		0 & \quad \text{otherwise}
	\end{array} \right.	
\end{align}
where $u_0 = \SI{10}{\meter\per\second}$, $z_1 = \SI{4}{\kilo\meter}$ and $z_2 = \SI{5}{\kilo\meter}$.
This results in a constant wind aloft, and zero flow at \SI{4}{\kilo\meter} and below.
A scalar anomaly $\varphi$ is positioned upstream above the height of the terrain.  It has the shape
\begin{align}
	\varphi(x, z) &= \varphi_0 \left\{ \begin{array}{l l}
		\cos^2 \left( \frac{\pi r}{2} \right) & \quad \text{if $r \leq 1$} \\
		0 & \quad \text{otherwise}
	\end{array} \right.
%
\intertext{having radius $r$ given by}
%
	r &= \sqrt{
		\left( \frac{x - x_0}{A_x} \right)^2 + 
		\left( \frac{z - z_0}{A_z} \right)^2
	}
\end{align}
where $A_x = \SI{25}{\kilo\meter}$, $A_z = \SI{3}{\kilo\meter}$ are the horizontal and vertical half-widths respectively, and $\varphi_0 = 1$ is the maximum magnitude of the anomaly.  At $t = \SI{0}{\second}$, the anomaly is centred at $(x_0, z_0) = (\SI{-50}{\kilo\meter}, \SI{9}{\kilo\meter})$ so that the anomaly is upwind of the mountain and well above the maximum terrain height of \SI{3}{\kilo\meter}.

\begin{figure}
	\centerfloat
	\input{advection-initial-plot}
	\caption{Vertical cross section of the two-dimensional advection test showing the horizontal wind profile, surface terrain profile and scalar anomaly field at $t = \SI{0}{\second}$ on a $\SI{300}{\kilo\meter} \times \SI{25}{\kilo\meter}$ domain.  Adapted from \textcite{schaer2002}.}
	\label{fig:advection:initial}
\end{figure}

The OpenFOAM solver \shellcmd{scalarTransportFoam} was used to implicitly solve the advection equation in flux form
\begin{align}
	\frac{\partial \varphi}{\partial t} + \del \cdot \left( \vect{u} \varphi \right) = 0
\end{align}
where $\vect{u}$ is the vector wind field.

The domain is discretised onto a grid having $300 \times 50$ cells such that $\Delta x = \SI{1}{\kilo\meter}$ and $\Delta \trans{z} = \SI{500}{\meter}$.
\TODO{boundary conditions -- zero gradient on T and U for all patches? -- unlike Schaer's periodic conditions on lateral boundary}
Tests are integrated forward in time for \SI{10000}{\second} with a timestep $\Delta t = \SI{25}{\second}$.  \TODOsidenote{Schaer's timestep is 25s, should we try to match that?}  \TODO{horiz/vertical courant number?}



\TODO{
\begin{itemize}
\item Compared meshes, and vanLeer versus tvdLimitedCubicUpwindCPCFit
\item Looked at l2 error norm at t=10000s by comparing with analytic tracer (created with setScalarOverOrography)
\item Cut-cell style meshes, unsurprisingly, same l2 error as noOrography
\item tvdLimitedCubicUpwindCPCFit slightly better than vanLeer, but much more computationally expensive
\end{itemize}
}
