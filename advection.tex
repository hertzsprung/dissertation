\section{Advection}
Following \textcite{schaer2002}, a scalar tracer is transported above the orography by solving the advection equation for a prescribed horizontal wind.
The wind profile, terrain profile and initial tracer field are shown in Figure~\ref{fig:advection:initial}.

\subsection{Specification}
The domain is \SI{300}{\kilo\meter} wide and \SI{25}{\kilo\meter} high.  The terrain is wave-shaped, specified by the surface height $\surface$ such that
\begin{align}
	\surface(x) &= \cos^2 \left( \frac{\pi x}{\lambda} \right) \surface^\star
%
	\intertext{where}
%
	\surface^\star(x) &= \left\{ \begin{array}{l l}
		h_0 \cos^2 \left( \frac{\pi x}{2a} \right) & \quad \text{if $| x | < a$} \\
		0 & \quad \text{otherwise}
	\end{array} \right.
\end{align}
where $a = \SI{25}{\kilo\meter}$ is the mountain half-width, $h_0 = \SI{3}{\kilo\meter}$ is the maximum mountain height, and $\lambda = \SI{8}{\kilo\meter}$ is the wavelength.  On the SLEVE grid, the large-scale component $\surface_1$, as described in section~\ref{sec:theory:sleve}, is given by
\begin{align}
	\surface_1(x) &= \frac{1}{2}\surface^\star(x)
\end{align}
For comparison, the same tests were performed with no orography, such that $\surface = \SI{0}{\kilo\meter}$ everywhere.

The wind is entirely horizontal and is prescribed as
\begin{align}
	u(z) = u_0 \left\{ \begin{array}{l l}
		1 & \quad \text{if $z \geq z_2$} \\
		\sin^2 \left( \frac{\pi}{2} \frac{z - z_1}{z_2 - z_1} \right) & \quad \text{if $z_1 < z < z_2$} \\
		0 & \quad \text{otherwise}
	\end{array} \right.	
\end{align}
where $u_0 = \SI{10}{\meter\per\second}$, $z_1 = \SI{4}{\kilo\meter}$ and $z_2 = \SI{5}{\kilo\meter}$.
This results in a constant wind aloft, and zero flow at \SI{4}{\kilo\meter} and below.
A tracer $\varphi$ is positioned upstream above the height of the terrain.  It has the shape
\begin{align}
	\varphi(x, z) &= \varphi_0 \left\{ \begin{array}{l l}
		\cos^2 \left( \frac{\pi r}{2} \right) & \quad \text{if $r \leq 1$} \\
		0 & \quad \text{otherwise}
	\end{array} \right.
%
\intertext{having radius $r$ given by}
%
	r &= \sqrt{
		\left( \frac{x - x_0}{A_x} \right)^2 + 
		\left( \frac{z - z_0}{A_z} \right)^2
	}
\end{align}
where $A_x = \SI{25}{\kilo\meter}$, $A_z = \SI{3}{\kilo\meter}$ are the horizontal and vertical half-widths respectively, and $\varphi_0 = 1$ is the maximum magnitude of the anomaly.  At $t = \SI{0}{\second}$, the anomaly is centred at $(x_0, z_0) = (\SI{-50}{\kilo\meter}, \SI{9}{\kilo\meter})$ so that the anomaly is upwind of the mountain and well above the maximum terrain height of \SI{3}{\kilo\meter}.  Analytic solutions can be found by adjusting the anomaly centre such that $x_0 = ut$.

\begin{figure}
	\centerfloat
	\input{advection-initial-plot}
	\caption{Vertical cross section of the two-dimensional advection test showing the horizontal wind profile, surface terrain profile and scalar tracer field at $t = \SI{0}{\second}$ on a $\SI{300}{\kilo\meter} \times \SI{25}{\kilo\meter}$ domain.  Adapted from \textcite{schaer2002}.}
	\label{fig:advection:initial}
\end{figure}

\subsection{Discretisation}
The OpenFOAM solver \shellcmd{scalarTransportFoam} was used to implicitly solve the advection-diffusion equation in flux form
\begin{align}
	\frac{\partial \varphi}{\partial t} + \del \cdot \left( \vect{u} \varphi \right) - \laplace \left( \kappa \varphi \right) = 0
\end{align}
where $\vect{u}$ is the vector wind field.  Diffusion was turned off by setting the diffusion coefficient $\kappa$ to be zero.

The domain is discretised onto a grid having $300 \times 50$ cells such that $\Delta x = \SI{1}{\kilo\meter}$ and $\Delta \trans{z} = \SI{500}{\meter}$.  Unlike \textcite{schaer2002} which used periodic lateral boundaries, we use zero gradient boundary conditions for all boundaries.
Tests are integrated forward in time for \SI{10000}{\second} with a timestep $\Delta t = \SI{25}{\second}$.  \TODOsidenote{we never looked at courant numbers, but Schaer talks about horiz/vertical Co numbers}

\TODO{describe gauss, vanLeer and tvdLimitedCubicUpwindCPCFit}

Two advection schemes were tested.

The second is an upwind-biased cubic interpolation from \cite{weller2009}.  \TODO{more explanation...}

\subsection{Results}
\begin{figure}
	\captionsetup[subfigure]{position=b}
	\centering
	\subcaptionbox{BTF \label{fig:advection:vanLeer:btf}}[0.48\textwidth]{\input{advection-btf-schaerCos-vanLeer-contour-plot}}
	\hfill
	\subcaptionbox{BTF \label{fig:advection:schaer:btf}}[0.48\textwidth]{\vspace{0.43in}\includegraphics[height=1.2in]{img/schaer-btf-centred.png}}
\\
	\subcaptionbox{SLEVE \label{fig:advection:vanLeer:sleve}}[0.48\textwidth]{\input{advection-sleve-schaerCos-vanLeer-contour-plot}}
	\hfill
	\subcaptionbox{SLEVE \label{fig:advection:schaer:sleve}}[0.48\textwidth]{\vspace{0.43in}\includegraphics[height=1.2in]{img/schaer-sleve-centred.png}}
\\
	\subcaptionbox{Cut cell \label{fig:advection:vanLeer:cutCell}}[0.48\textwidth]{\input{advection-snapCol-schaerCos-vanLeer-contour-plot}}
	\hfill
	\subcaptionbox{Analytic solution on a regular grid}[0.48\textwidth]{\TODO{}}
%
	\caption{\TODO{van Leer advection compared with schaer.  Contour intervals every 0.1}}
	\label{fig:advection:vanLeer}
\end{figure}

Results of advection using the van Leer scheme are presented in Figure~\ref{fig:advection:vanLeer}.  On the BTF grid, the tracer suffers from distortion over the mountain and some artefacts just above the mountain remain as the tracer moves over it.  Compared with the results from \textcite{schaer2002}, the tracer retains more of its shape but is far more diffusive by the end of the simulation, as evidenced by the missing contours in the tracer centre.

Results on the SLEVE grid are much closer to the analytic solution on a regular grid.  The tracer retains its shape throughout the simulation and does not suffer from any noticeable distortion.  However, the widening contours indicate some diffusion is still present which, again, was not found in the results of \textcite{schaer2002}.

Since the cut cell grid is entirely regular away from the surface, it is unsurprising that the results are the same as advection on a regular grid (not shown).

\TODO{now discuss tvdLimitedCubicUpwindCPCFit}

Error norms were calculated at $t = \SI{10000}{\second}$ by comparing with the analytic solution.  The $\ell^2$ error norm is defined as
\begin{align}
\ell^2 = \sqrt{\frac{\sum \left( \varphi - \varphi_T \right)^2 \cdot A}{\sum A}}
\end{align}
where $\varphi$ is the numerical tracer value, $\varphi_T$ is the analytical value and $A$ is the cell area.  The resulting errors are summarised in table~\ref{tab:advection:errors}.  \TODO{what do they tell us?}

\begin{table}
\centering
\begin{tabular}{ c @{\hspace{2em}} c c}
\toprule
		&	van Leer	& Cubic upwind \\ \midrule
BTF		& 0			& 0 \\
SLEVE		& 0			& 0 \\
Cut cell	& 0			& 0 \\
Regular grid	& 0			& 0 \\ \bottomrule
\end{tabular}

\caption{$\ell^2$ error norms for the advection test at $t = \SI{10000}{\second}$}
\label{tab:advection:errors}
\end{table}

\TODO{
\begin{itemize}
\item Compared meshes, and vanLeer versus tvdLimitedCubicUpwindCPCFit
\item Looked at l2 error norm at t=10000s by comparing with analytic tracer (created with setScalarOverOrography)
\item Cut-cell style meshes, unsurprisingly, same l2 error as noOrography
\item tvdLimitedCubicUpwindCPCFit slightly better than vanLeer, but much more computationally expensive
\end{itemize}
}
