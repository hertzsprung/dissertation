\section{Horizontal tracer advection}
\label{sec:advection}

Following \textcite{schaer2002}, a tracer is transported above orography by solving the advection equation for a prescribed horizontal wind.  This challenges the accuracy of the advection scheme in the presence of grid distortions.
The wind profile, terrain profile and initial tracer field are shown in Figure~\ref{fig:advection:initial}.

\subsection{Specification}
The domain is \SI{300}{\kilo\meter} wide and \SI{25}{\kilo\meter} high.  The terrain is wave-shaped, specified by the surface height $\surface$ such that
\begin{subequations}
\label{eqn:advection:schaerCos}
\begin{align}
	\surface(x) &= \cos^2 \left( \frac{\pi x}{\lambda} \right) \surface^\star
%
	\intertext{where}
%
	\surface^\star(x) &= \left\{ \begin{array}{l l}
		h_0 \cos^2 \left( \frac{\pi x}{2a} \right) & \quad \text{if $| x | < a$} \\
		0 & \quad \text{otherwise}
	\end{array} \right.
\end{align}
\end{subequations}
where $a = \SI{25}{\kilo\meter}$ is the mountain half-width, $h_0 = \SI{3}{\kilo\meter}$ is the maximum mountain height, and $\lambda = \SI{8}{\kilo\meter}$ is the wavelength.  On the SLEVE grid, the large-scale component $\surface_1$, as described in section~\ref{sec:theory:tf}, is given by
\begin{align}
	\surface_1(x) &= \frac{1}{2}\surface^\star(x)
\end{align}
and $s_1 = \SI{15}{\kilo\meter}$ is the large scale height, and $s_2 = \SI{2.5}{\kilo\meter}$ is the small scale height.  The optimisation of SLEVE by \textcite{leuenberger2010} is not used, so the exponent $n = 1$.

For comparison, the same tests were performed with no orography, such that $\surface = \SI{0}{\kilo\meter}$ everywhere.

The wind is entirely horizontal and is prescribed as
\begin{align}
	u(z) = u_0 \left\{ \begin{array}{l l}
		1 & \quad \text{if $z \geq z_2$} \\
		\sin^2 \left( \frac{\pi}{2} \frac{z - z_1}{z_2 - z_1} \right) & \quad \text{if $z_1 < z < z_2$} \\
		0 & \quad \text{otherwise}
	\end{array} \right.	
\end{align}
where $u_0 = \SI{10}{\meter\per\second}$, $z_1 = \SI{4}{\kilo\meter}$ and $z_2 = \SI{5}{\kilo\meter}$.
This results in a constant wind aloft, and zero flow at \SI{4}{\kilo\meter} and below.
A tracer $\varphi$ is positioned upstream above the height of the terrain.  It has the shape
\begin{align}
	\varphi(x, z) &= \varphi_0 \left\{ \begin{array}{l l}
		\cos^2 \left( \frac{\pi r}{2} \right) & \quad \text{if $r \leq 1$} \\
		0 & \quad \text{otherwise}
	\end{array} \right.
%
\intertext{having radius $r$ given by}
%
	r &= \sqrt{
		\left( \frac{x - x_0}{A_x} \right)^2 + 
		\left( \frac{z - z_0}{A_z} \right)^2
	}
\end{align}
where $A_x = \SI{25}{\kilo\meter}$, $A_z = \SI{3}{\kilo\meter}$ are the horizontal and vertical half-widths respectively, and $\varphi_0 = 1$ is the maximum magnitude of the anomaly.  At $t = \SI{0}{\second}$, the anomaly is centred at $(x_0, z_0) = (\SI{-50}{\kilo\meter}, \SI{9}{\kilo\meter})$ so that the anomaly is upwind of the mountain and well above the maximum terrain height of \SI{3}{\kilo\meter}.  Analytic solutions can be found by adjusting the anomaly centre such that $x_0 = ut$.

\begin{figure}
	\centerfloat
	\input{advection-initial-plot}
	\caption{Vertical cross section of the two-dimensional advection test showing the horizontal wind profile, surface terrain profile and tracer field at $t = \SI{0}{\second}$ on a $\SI{300}{\kilo\meter} \times \SI{25}{\kilo\meter}$ domain.  Adapted from \textcite{schaer2002}.}
	\label{fig:advection:initial}
\end{figure}

\subsection{Discretisation}
The OpenFOAM solver \shellcmd{scalarTransportFoam} was used to implicitly solve the advection equation in flux form
\begin{align}
	\frac{\partial \varphi}{\partial t} + \del \cdot \left( \vect{u} \varphi \right) = 0 \label{eq:advection:continuous}
\end{align}

The time derivative is solved implicitly using a backward-in-time, second order accurate scheme defined as \autocite{openfoam-progguide}.  At time level $n$, the time derivative, $\partial \varphi / \partial t$, is
\begin{align}
	\frac{\partial}{\partial t} \int_V \varphi \diff V = \frac{
		3 \left( \varphi V \right)^{(n)} - 
		4 \left( \varphi V \right)^{(n-1)} + 
		\left( \varphi V \right)^{(n-2)}
	}{2 \Delta t}
\end{align}

Spatial discretisation follows the finite volume method described in section~\ref{sec:theory:fv}, using the upwind-biased advection scheme described in section~\ref{sec:method:discretisation}.
The domain is discretised onto a grid having $300 \times 50$ cells such that $\Delta x = \SI{1}{\kilo\meter}$ and $\Delta \trans{z} = \SI{500}{\meter}$.  Unlike \textcite{schaer2002} who use periodic lateral boundaries, we use a fixed value of 0 at the inlet boundary and zero gradient boundaries elsewhere.
Tests are integrated forward in time for \SI{10000}{\second} with a timestep $\Delta t = \SI{25}{\second}$.

\subsection{Results}
\begin{figure}
	\captionsetup[subfigure]{position=b}
	\centering
	\subcaptionbox{BTF (negative contours at $t = \SI{10000}{\second}$ near mountain peak shown as dashed red lines) \label{fig:advection:cubicUpwind:btf}}[0.49\textwidth]{\input{advection-btf-schaerCos-cubicUpwindCPCFit-contour-plot}}
	\hfill
	\subcaptionbox{BTF from \textcite{schaer2002} (negative contours shown as dashed lines) \label{fig:advection:schaer:btf}}[0.49\textwidth]{\vspace{0.43in}\includegraphics[height=1.2in]{img/schaer-btf-centred.png}}
\\
	\subcaptionbox{SLEVE \label{fig:advection:cubicUpwind:sleve}}[0.49\textwidth]{\input{advection-sleve-schaerCos-cubicUpwindCPCFit-contour-plot}}
	\hfill
	\subcaptionbox{SLEVE from \textcite{schaer2002} \label{fig:advection:schaer:sleve}}[0.49\textwidth]{\vspace{0.43in}\includegraphics[height=1.2in]{img/schaer-sleve-centred.png}}
\\
	\subcaptionbox{SnapCol grid \label{fig:advection:cubicUpwind:snapCol}}[0.49\textwidth]{\input{advection-snapCol-schaerCos-cubicUpwindCPCFit-contour-plot}}
	\hfill
	\subcaptionbox{Analytic solution on a regular grid \label{fig:advection:analytic}}[0.49\textwidth]{\input{advection-noOrography-analytic-contour-plot}}
%
	\caption{Horizontally advected tracer contours at $t = \SI{0}{\second}$, \SI{5000}{\second} and \SI{10000}{\second}.  Figures (\protect\subref{fig:advection:cubicUpwind:btf}), (\protect\subref{fig:advection:cubicUpwind:sleve}), and (\protect\subref{fig:advection:cubicUpwind:snapCol}) use the upwind-biased scheme described in section~\ref{sec:method:discretisation}.  Figures (\protect\subref{fig:advection:schaer:btf}) and (\protect\subref{fig:advection:schaer:sleve}) show the results of the second-order centred difference scheme from \textcite{schaer2002}.  Contour intervals are every 0.1.}
	\label{fig:advection:cubicUpwind}
\end{figure}

\begin{figure}
	\captionsetup[subfigure]{position=b}
	\centering
	\subcaptionbox{BTF \label{fig:advection:error:btf:cubicUpwind}}[0.49\textwidth]{\includegraphics[width=1.3in,angle=270]{openfoam/cases/advection/btf/schaerCos/cubicUpwindCPCFit/10000/tracer-contour-error.eps}}
	\hfill
	\subcaptionbox{BTF from \textcite{schaer2002} \label{fig:advection:error:schaer:btf}}[0.49\textwidth]{\vspace{0.1in}\includegraphics[height=1.2in]{img/schaer-btf-centred-error.png}} \\
%
	\subcaptionbox{SLEVE \label{fig:advection:error:sleve:cubicUpwind}}[0.49\textwidth]{\includegraphics[width=1.3in,angle=270]{openfoam/cases/advection/sleve/schaerCos/cubicUpwindCPCFit/10000/tracer-contour-error.eps}}
	\hfill
	\subcaptionbox{SLEVE from \textcite{schaer2002} \label{fig:advection:error:schaer:sleve}}[0.49\textwidth]{\vspace{0.1in}\includegraphics[height=1.2in]{img/schaer-sleve-centred-error.png}} \\
%
	\subcaptionbox{Regular grid \label{fig:advection:error:noOrography:cubicUpwind}}[0.49\textwidth]{\includegraphics[width=1.3in,angle=270]{openfoam/cases/advection/noOrography/cubicUpwindCPCFit/10000/tracer-contour-error.eps}}
	\hfill
	\subcaptionbox{Regular grid from \textcite{schaer2002} \label{fig:advection:error:schaer:noOrography}}[0.49\textwidth]{\vspace{0.1in}\includegraphics[height=1.2in]{img/schaer-noOrography-centred-error.png}} \\
%
	\caption{Errors in horizontal tracer advection at $t = \SI{10000}{\second}$.  Figures (\protect\subref{fig:advection:error:btf:cubicUpwind}), (\protect\subref{fig:advection:error:sleve:cubicUpwind}) and (\protect\subref{fig:advection:error:noOrography:cubicUpwind}) use the upwind-biased scheme.  Figures (\protect\subref{fig:advection:error:schaer:btf}), (\protect\subref{fig:advection:error:schaer:sleve}) and (\protect\subref{fig:advection:error:schaer:noOrography}) show the error of the second-order centred difference scheme from \textcite{schaer2002}.  Contour intervals are every 0.01, with negative contours denoted by dashed lines.}
	\label{fig:advection:error}
\end{figure}

Results of advection are presented in figure~\ref{fig:advection:cubicUpwind}.
On the BTF grid, the tracer suffers from distortion over the mountain and some artefacts just above the mountain remain as the tracer moves over it.  Comparing figures~\ref{fig:advection:cubicUpwind:btf} and \ref{fig:advection:schaer:btf}, we see that the tracer retains its shape far better than the result from \textcite{schaer2002} that uses a second-order centred difference scheme.  This is expected since the upwind-biased cubic scheme has a larger stencil and higher order accuracy.  Comparing figures~\ref{fig:advection:error:btf:cubicUpwind} and \ref{fig:advection:error:schaer:btf} we see that, unlike the results from \textcite{schaer2002}, errors on the BTF grid are confined to regions around the tracer and near the mountain peak.

As seen in figure~\ref{fig:advection:cubicUpwind:sleve}, results on the SLEVE grid are much closer to the analytic solution on a regular grid (figure~\ref{fig:advection:analytic}).  The tracer retains its shape throughout the simulation and does not suffer from any noticeable distortion.  We find that accuracy is slightly better than the result from \textcite{schaer2002} (see figures~\ref{fig:advection:error:sleve:cubicUpwind} and \ref{fig:advection:error:schaer:sleve}).  Unlike \textcite{schaer2002}, a further improvement in accuracy is seen on a regular grid, as shown in figure~\ref{fig:advection:error:noOrography:cubicUpwind}.

Since the SnapCol grid is entirely regular away from the surface, it is unsurprising that the results (shown in figure~\ref{fig:advection:cubicUpwind:snapCol}) are the same as advection on a regular grid (not shown).  This result agrees with that found by \textcite{good2013}.

At $t = \SI{0}{\second}$, the tracer ranges between zero and one.  Over time, new extrema are generated because the upwind-biased advection scheme is not monotonic.  This is most evident on the BTF grid where the stationary artefacts above the mountain peak reach a minimum of \input{openfoam/cases/advection/btf/schaerCos/cubicUpwindCPCFit/min.txt} by the end of the simulation.  The results of the second-order centred difference scheme of \textcite{schaer2002} show significant negative tracer values as evidenced by the dashed contours in figure~\ref{fig:advection:schaer:btf}.  Minimum values remain close to zero on the SLEVE, SnapCol and regular grids.  All grids show some decrease in maximum tracer magnitude and, once again, the decrease is most severe on the BTF grid.  Results of tracer extrema on all grids are compared to the analytic solution in figure~\ref{fig:wobblyTracerAdvection:ranges:horizontal} and the values are given in table~\ref{tab:advection:errors}.

Error norms were calculated at $t = \SI{10000}{\second}$ by comparing with the analytic solution.  The $\ell^2$ error norm is defined as
\begin{align}
\ell^2 = \sqrt{\frac{\sum \left( \varphi - \varphi_T \right)^2 V}{\sum V}}
\end{align}
where $\varphi$ is the numerical tracer value, $\varphi_T$ is the analytic value and $V$ is the cell volume.  Because the test is two dimensional, the cell volume is equivalent to the cell area.  The resulting errors are summarised in table~\ref{tab:advection:errors}.  Errors on the BTF grid are an order of magnitude greater than the three other grids tested.  The cut cell grid offers only a small error reduction compared to the SLEVE grid.  Even on the BTF grid, the upwind-biased advection scheme is far more tolerant of grid distortions than results of the fourth-order centred scheme from \textcite{schaer2002} (not shown).

\TODO{why do we get massive errors over the mountain?  plot tracer contours at t=5000s including the grid lines}

