A variety of methods exist to represent mountains in atmospheric models.  The most common method in use today is that of terrain following layers.  Grids having terrain following layers can be straightfowardly stored in array structures which simplifies parallel computation.
However, terrain following grids have the disadvantage that, as model resolution increases, gradients in terrain tend to become steeper, which can lead to greater numerical errors.

An alternative to terrain following layers is the cut cell method.  Cut cell grids are better able to represent steep slopes, but can result in very small cells that limit the model timestep unless the grid or the discretisation is modified to account for them.
While each method offers potential advantages, there is little consensus about which method offers the most accurate results for numerical weather prediction.

This project assesses the accuracy of terrain following and cut cell grids using a collection of two-dimensional test cases, including newly developed, terrain following advection tests.  Two simulations are performed using a nonhydrostatic model and results are compared to those from the literature.  First, spurious flow is analysed for an atmosphere at rest in the presence of an idealised mountain profile.  Second, gravity waves are forced by horizontal flow over idealised orography.

In addition, a series of advection tests are performed that challenge the accuracy of the model's advection scheme.  First, a tracer is advected above orography in a horizontal velocity field.
Second, a new test is developed to tax the advection scheme on the cut cell grid by advecting a tracer in a velocity field that follows the terrain.
The third test is constructed to investigate the source of thermal errors found on the cut cell grid in the gravity waves test.  Here, the same thermal profile is advected using another terrain following velocity field.

We find that advection scheme accuracy is greatest when advection follows grid layers.  Advection is accurate in horizontal flows on cut cell grids since the grid is uniform aloft.  There are modest errors on terrain following grids, but results are nonetheless more accurate than those from the literature.  Results are accurate for terrain following flows on terrain following grids, but errors are significant on cut cell grids.

Spurious velocities are reduced in a resting atmosphere by using cut cell grid.  However, in the gravity waves test, we find errors in potential temperature on the cut cell grid that are the result of the Lorenz computational mode.   By advecting the same thermal profile in a terrain following flow, thermal errors are once again found in the same position on the cut cell grid.  However, the anomalies are of opposite signs compared to those in the gravity waves test.

The cause is of the thermal errors in the gravity waves test is not yet certain, but may be due to errors in the advection of potential temperature, or an error in the velocity field itself.  The velocity field used in the thermal advection test is not identical to the velocity field in the gravity waves test, and this may be the reason for the reversal of potential temperature anomalies.
