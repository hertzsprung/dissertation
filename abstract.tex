A variety of methods exist to represent mountains in atmospheric models.  The most common method in use today is that of terrain following layers.  However, terrain following grids have the problem that, as model resolution increases, gradients in terrain tend to become steeper, which can lead to greater numerical errors.

An alternative to terrain following layers is the cut cell method.  Cut cell grids are better able to represent steep slopes, but can result in very small cells that limit the model timestep unless the grid or the discretisation is modified to account for them.
While each method offers potential advantages, cut cell grids have been found to give more accurate results in standard test cases from the literature.

This project assesses the accuracy of terrain following and cut cell grids using a collection of two-dimensional test cases.  Two simulations are performed using a nonhydrostatic model and results are compared to those from the literature.  First, spurious flow is analysed for an atmosphere at rest in the presence of an idealised mountain profile.  Second, gravity waves are forced by horizontal flow over idealised orography.  In addition, a series of advection tests are performed that challenge the accuracy of the model's advection scheme.  These include newly developed tests that are designed to challenge accuracy on cut cell grids.

We find that advection scheme accuracy is greatest when advection follows grid layers.  Advection is accurate in horizontal flows on cut cell grids since the grid is uniform aloft.  There are modest errors on terrain following grids, but results are nonetheless more accurate than those from the literature.  Results are accurate for terrain following flows on terrain following grids, but errors are significant on cut cell grids.

Spurious velocities are reduced in a resting atmosphere by using cut cell grid.  However, in the gravity waves test, we find errors in potential temperature on the cut cell grid that are associated with the Lorenz computational mode.

We confirm results from previous studies that show cut cells give good accuracy in certain test cases with flow or stratification that is aligned with the grid.  However, we find that terrain following grids give reasonable accuracy in all test cases.  Where flow interacts more significantly with the orography, as found in tests of gravity waves and terrain following advection, results are more accurate on terrain following grids than with cut cells.
