\chapter{Theoretical basis}

\section{Coordinate transformations}
Consider the two-dimensional cartesian coordinates $(x, z)$ and transformed coordinates $(x, \trans{z})$ which must be a monotonic function of $z$ so that the transformation is invertible.

A scalar field $\varphi$ can be expressed in transformedcoordinates as $\varphi(x, \trans{z})$ or in cartesian coordinates as $\varphi(x, \trans{z}(x, z))$.
Hence the vertical derivative of $\varphi$ can be found by using the chain rule
\begin{align}
  \frac{\partial \varphi}{\partial z} =
  \frac{\partial \varphi}{\partial \trans{z}}
  \frac{\partial \trans{z}}{\partial z}
\end{align}

The multivariable chain rule is needed to find the horizontal derivative.  Given a $m \times n$ Jacobi rotation matrix
\begin{align}
\frac{\partial y_i}{\partial x_j} = 
\begin{bmatrix}
  \dfrac{\partial y_1}{\partial x_1}	& \dfrac{\partial y_1}{\partial x_2} &	\cdots &	\dfrac{\partial y_1}{\partial x_n} \\
  \vdots				& \vdots &				\ddots &	\vdots \\
  \dfrac{\partial y_m}{\partial x_1}	& \dfrac{\partial y_m}{\partial x_2} &	\cdots &	\dfrac{\partial y_m}{\partial x_n}
\end{bmatrix}
%
\intertext{we can express the chain rule as}
%
\frac{\partial y_i}{\partial x_j} = \frac{\partial y_i}{\partial u_i} \frac{\partial u_i}{\partial x_j}
%
\intertext{\TODO{explain that these are matrices.  would be better to use vectors where possible}  Applying this to $\varphi$ we find}
%
\begin{bmatrix}
	\frac{\partial \varphi}{\partial x}  &  \frac{\partial \varphi}{\partial z}
\end{bmatrix}
=
\begin{bmatrix}
	\frac{\partial \varphi}{\partial x}  &  \frac{\partial \varphi}{\partial \trans{z}}
\end{bmatrix}
\begin{bmatrix}
	\frac{\partial x}{\partial x} & 	\frac{\partial x}{\partial z} \\
	\frac{\partial \trans{z}}{\partial x} &	\frac{\partial \trans{z}}{\partial z}
\end{bmatrix}
%
\intertext{\TODO{the horizontal component of the previous equation is...}}
%
\left. \frac{\partial \varphi}{\partial x} \right|_z =
\left. \frac{\partial \varphi}{\partial x} \right|_\trans{z} +
	\frac{\partial \varphi}{\partial \trans{z}}
	\left. \frac{\partial \trans{z}}{\partial x} \right|_z \label{eq:theory:coord-trans}
\end{align}
where the subscript by the vertical bar denotes the variable that is held constant.  \TODOsidenote{This came from \href{http://ocw.mit.edu/courses/earth-atmospheric-and-planetary-sciences/12-950-atmospheric-and-oceanic-modeling-spring-2004/lecture-notes/lec25.pdf}{MIT lecture notes} -- I'm not sure if I need to cite them or what.}

In a two-dimensional terrain-following coordinate transform, there is the added requirement that the transformed domain be rectangular.  This can be satisfied by imposing boundary conditions \autocite{schaer2002}
\begin{align}
	\trans{z}(x, \surface(x)) = 0 \quad,\quad \trans{z}(x, H) = H
\end{align}
where $H$ is the height of the domain and $h(x)$ is the height of the terrain surface.

\section{Horizontal pressure gradient}
\begin{figure}
	\caption{\TODO{pressure gradient error (adapt from Mahrer)}}
	\label{fig:theory:pressure-error}
\end{figure}

It follows from equation~\ref{eq:theory:coord-trans} that, in TF coordinates, the horizontal gradient of pressure $p$ is \autocite{mahrer1984}
\begin{align}
	\left. \frac{\partial p}{\partial x} \right|_z = 
	\left. \frac{\partial p}{\partial x} \right|_\trans{z} + 
	\left. \frac{\partial \trans{z}}{\partial x} \right|_z
	\frac{\partial p}{\partial \trans{z}}
\end{align}
The first term on the right hand side is the change in pressure along the TF coordinate surface, and the second term corrects for the vertical variation in the first.  These terms tend to be large and of opposite sign over steep terrain, and care must be taken \TODO{lame.  say that ideally you want complete cancellation, but not possible numerically.  at least, metric terms must be consistent (i.e. at least first order accurate)}  that the hydrostatic components cancel in the discretisation of the two terms \autocite{gary1973}.  Errors in the horizontal pressure gradient are associated with horizontal acceleration by the momentum equation, and have been shown to generate spurious winds \parencites{klemp2003}{klemp2011}.

\section{Grids}
\subsection{Basic terrain following (BTF)}
\label{sec:theory:btf}

The basic terrain following (BTF) coordinates of \textcite{galchen-somerville1975} are defined as
\begin{align}
	\trans{z} &= H \frac{z - \surface}{H - \surface}
%
\intertext{or}
%
	z &= \left( H - \surface \right) \frac{\trans{z}}{H} + \surface
\end{align}
where, in two dimensions, $z(x, \trans{z})$ is the height of the coordinate surface at level $\trans{z}$, $H$ is the height of the domain, and $\surface(x)$ is the height of the terrain surface.
\TODOsidenote{BTF and sigma are equivalent in their transform.  BTF is height, sigma is pressure.  however, pressure varies horizontally}

\subsection{Smooth level vertical (SLEVE)}
\label{sec:theory:sleve}
Unlike most TF coordinates in which the vertical transformation depends only on terrain height, \textcite{schaer2002} describes a coordinate system in which vertical decay is scale-dependent.  Terrain height is split into a large-scale component $\surface_1$ and a small-scale component $\surface_2$ such that $\surface = \surface_1 + \surface_2$.  The transformation is defined as
\begin{align}
	z &= \trans{z} + \surface_1 b_1 + \surface_2 b_2
%
\intertext{with the vertical decay functions are given by}
%
	b_i &= \frac{\sinh \left( \left( H - \trans{z} \right) / s_i \right)}{\sinh \left( H / s_i \right)}
\end{align}
where $s_1$ and $s_2$ are the scale heights of large-scale and small-scale terrain respectively.

\textcite{leuenberger2010} generalised the SLEVE transformation in order to increase cell thickness in the layers nearest the ground.  An exponent $n$ is introduced so that the generalised decay functions become
\begin{align}
	b_i &= \frac{\sinh \left( \left( H / s_i \right)^n - \left( \trans{z} / s_i \right)^n \right)}{\sinh \left( H / s_i \right)^n}
\end{align}
where the optimal exponent value was found to be $n = 1.35$.

\subsection{Cut cell}
\TODO{as set out by rosatti 2005, klein 2009 etc}

The cut cell method intersects the surface with a regular cartesian grid.  This leads to cells that are entirely above the surface, entirely below it, and those which intersect with the ground.  Those cells which intersect the ground have, in two dimensions, a triangular, trapezoidal or pentagonal shape\autocite{rosatti2005}.

\TODO{\textbf{finite volume technique} (adcroft 1997 has some nice stuff)  relevant to cut cells and, more generally, because all our work is in cartesian coord sys and OpenFOAM is all FV}

\TODO{\textbf{CFL criterion}.  relevant to understand timestep constraints on small cells}

\TODO{\textbf{TVD}}

