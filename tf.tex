\section{Terrain following techniques}
\label{sec:theory:tf}

\begin{figure}
	\captionsetup[subfigure]{position=b}
	\centering
	\subcaptionbox{Basic terrain following (BTF) \label{fig:intro:tf:btf}}[0.45\textwidth]{\input{btf-plot}}
	\hfill
	\subcaptionbox{Smooth level vertical (SLEVE) \label{fig:intro:tf:sleve}}[0.45\textwidth]{\input{sleve-plot}}
	\caption{Example vertical cross sections of terrain following layers illustrating the decay in terrain influence with height.   In BTF the decay is linear; in SLEVE it is exponential.}
	\label{fig:intro:tf}
\end{figure}

As well as improving discretisation accuracy, errors due to coordinate transformation can also be reduced by smoothing the effect of the terrain so that the grid becomes more regular aloft. 

\textcite{galchen-somerville1975} proposed a basic terrain following (BTF) coordinate system in which the terrain's influence decays linearly with height but disappears only at the top of the domain (example shown in figure~\ref{fig:intro:tf:btf}).  The transformation is defined as
\begin{align}
	\trans{z} &= H \frac{z - \surface}{H - \surface} \label{eqn:theory:btf}
%
\intertext{or}
%
	z &= \left( H - \surface \right) \frac{\trans{z}}{H} + \surface
\end{align}
where, in two dimensions, $z(x, \trans{z})$ is the height of the coordinate surface at level $\trans{z}$, $H$ is the height of the domain, and $\surface(x)$ is the height of the terrain surface.

The sigma coordinate transform of \textcite{phillips1957} is equivalent to the BTF coordinate transform since they both decay linearly.  However, since they decay with pressure rather than height, sigma coordinates also change with horizontal variations in pressure.

The hybrid terrain following (HTF) coordinates of \textcite{simmons-burridge1981} improve upon BTF coordinates by allowing the vertical decay profile can be controlled.  By choosing a suitable profile, terrain influence decays more rapidly than BTF to produce surfaces of constant height aloft \autocite{klemp2011}.

The coordinate system can be further refined by decaying small-scale features more rapidly than large-scale features to produce smooth coordinate surfaces in the middle and top of the domain.
\textcite{schaer2002} achieved this with smooth level vertical (SLEVE) coordinates in which terrain height is split into a large-scale component $\surface_1$ and a small-scale component $\surface_2$ such that $\surface = \surface_1 + \surface_2$.  Each component has a different exponential decay profile.  The transformation is defined as
\begin{align}
	z &= \trans{z} + \surface_1 b_1 + \surface_2 b_2
%
\intertext{with the vertical decay functions are given by}
%
	b_i &= \frac{\sinh \left( \left( H - \trans{z} \right) / s_i \right)}{\sinh \left( H / s_i \right)}
\end{align}
where $s_1$ and $s_2$ are the scale heights of large-scale and small-scale terrain respectively.  SLEVE produces smooth coordinate surfaces in the middle and top of the domain (see figure~\ref{fig:intro:tf:sleve}).

\textcite{leuenberger2010} generalised the SLEVE transformation in order to increase cell thickness in the layers nearest the ground, allowing longer timesteps and permitting more accurate calculation of parameterised low-level heat and momentum fluxes.  An exponent $n$ is introduced so that the generalised decay functions become
\begin{align}
	b_i &= \frac{\sinh \left( \left( H / s_i \right)^n - \left( \trans{z} / s_i \right)^n \right)}{\sinh \left( H / s_i \right)^n}
\end{align}
where the optimal exponent value was found to be $n = 1.35$.

In the smoothed terrain following (STF) coordinate, \textcite{klemp2011} took an alternative approach, using a multipass smoothing operator, and found that errors were reduced still further compared to SLEVE.
