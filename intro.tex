\chapter{Introduction}
% why is orography important?
% explain the difference between coordinates and grids
% how is it currently represented in models?
% evolution of terrain following coordinates

%- traditional NWP, hydrostatic models, used pressure coords, varied over time (steppeler2002)
%- current nonhydrostatic models have geometric vertical coords constant in time (steppeler2012)
%---------------------

Orography has significant effects on local weather, creating strong downslope winds and enhancing local precipitation \autocite{TODO}.  Terrain also affects global circulation by exerting low-level drag \autocite{lottmiller1997} and transporting momentum via gravity waves \autocite{mcfarlane1987}.

To capture these effects, numerical weather prediction (NWP) models must represent the underlying orography.   

Traditional numerical weather prediction (NWP) models operate at resolutions where the hydrostatic balance approximation is valid and the use of pressure-based vertical coordinates simplify the equations \autocite{eliassen1949}.  TODO: talk about sigma coordinates briefly.  However, unlike their predecessors, current high resolution models deviate from hydrostatic balance because the aspect ratio between vertical and horizontal scales is too large \autocite{daley1988}.  Hence, most nonhydrostatic models use geometric vertical coordinates \autocite{steppeler2003}.

%TODO: Vertical motion is important... orographically induced winds (steppeler2002), buoyancy effects.  Hence a vertical coordinate/grid that minimises numerical errors is desirable. TODO: distinguish between coordinates and grids (metric terms etc)

Terrain following coordinates are in widespread use in operational nonhydrostatic models.  In these vertical, geometric coordinate systems the terrain's influence decays with height: the bottommost layers follow the underlying surface closely while the uppermost layers are flat.  Such coordinates are attractive because cell sizes remain almost constant (Jebens 2011) and terrain following grids are simple to represent in a rectangular data structure (Lock2012).  However, as horizontal model resolution increases, steeper orographic gradients induce spurious winds due to numerical errors. \autocite{TODO}  TODO: elaborate on how these errors arise.  These are truncation errors?  Janjic 1989

TODO: what has been done to reduce these errors?
% Klemp 2011 presents this nicely
Basic terrain following (BTF) \autocite{galchen1975} 
\begin{align}
z = \left( z_t - h \right) \frac{\zeta}{z_t} + h = \zeta + \left( 1 - \frac{\zeta}{z_t} \right) h
\end{align}
where TODO define $z$, $z_t$, $\zeta$ and $h$ is the terrain height.

HTF (Arakawa and Lamb 1977) (Simmons and Burridge 1981)
SLEVE
STF (Klemp 2011)

The cut cell (or `shaved cell')  method is an alternative to terrain following coordinates.  Here, the surface terrain is intersected with a regular Cartesian grid such that cells are cut where they intersect with the ground.  TODO: disadvantage: boundary layer parameterisations (Zaengl 2012)
