\section{Finite volume method}
\label{sec:theory:fv}

The finite volume method is a discretisation technique that represents the spatial domain as a non-overlapping grid of cells with fields represented as piecewise constant in each cell.  At each timestep, cell averages are updated by considering the flux $\vect{F}$ across the faces of the cell surface.  In atmospheric modelling, $\vect{F}$ is typically the advective flux $\vect{u} \varphi$ where $\vect{u}$ is the velocity field.

The notation used in this project for the representation of discrete variables follows \textcite{weller-shahrokhi2014}.  A cell average of $\varphi$ is written as $\varphi_c$, where $c$ denotes a cell.  A scalar field, $\psi$, located at a face is written as $\psi_f$, where $f$ denotes a cell face.  An interpolation of cell centre averages of $\varphi$ onto a face $f$ is written as $\varphi_F$.  $f \in \: c$ represents the faces of a cell.  $c \in \: f$ represents the cells $c$ that share a face $f$.

To describe the finite volume method, we start by considering the conservation of $\varphi$
\begin{align}
	\frac{\partial \varphi}{\partial t} + \del \cdot \left( \vect{u} \varphi \right) = 0
%
\intertext{Taking the volume integral over a cell $c$ with volume $V_c$ we find}
%
	\int_{V_c} \frac{\partial \varphi}{\partial t} \diff V + \int_{V_c} \del \cdot \left( \vect{u} \varphi \right) \diff V &= 0
%
\intertext{Integrating the first term gives the cell average $\varphi_c$ and applying the divergence theorem to the second term gives a surface integral, hence}
%
	V_c \frac{\partial \varphi_c}{\partial t} + \int_{S_c} \varphi \vect{u} \cdot \nunit \diff S &= 0
%
\intertext{where $S_c$ is the cell surface area and $\nunit$ is the unit vector that is outward normal to the surface.  Dividing by the cell volume and, for a cell with a finite number of surfaces, this becomes}
%
	\frac{\partial \varphi_c}{\partial t} + \frac{1}{V_c} \sum_{f \in \: c} \varphi_F \vect{u}_f \cdot \nunit S_f &= 0 \label{eqn:theory:discrete-continuity}
\end{align}
where $S_f$ is the surface area of face $f$, and $\varphi_F$ is the interpolated value of $\varphi$ at the face.  This interpolation of cell averages onto faces is a significant source of truncation error in finite volume systems \autocite{adcroft1997}.

The finite volume method often leads to a staggering of variables with fluxes at cell faces similar to an Arakawa C grid.  Because cell averages are modified only through surface fluxes, a quantity of $\varphi$ the fluxes out of one cell must flux into another.  Thus, $\varphi$ is conserved in the finite volume method.
