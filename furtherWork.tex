\chapter{Further work}

No \TODO{not none, some slight} evidence of the small cell problem was found in the gravity waves test (section~\ref{sec:gw}).  The Courant number was shown to be independent of horizontal velocity and, because flow is near-horizontal in the gravity waves test, this was why there was no small cell problem.  To confirm this hypothesis, another test might be constructed in which a cold thermal anomaly descends to the ground, similar to the rising bubble test used in several other studies \parencites{bonaventura2000}{jebens2011}{good2013}.  The vertical momentum that reaches the ground should, for a sufficiently large timestep, cause numerical instability on the SnapCol grid but a stable solution on terrain following grids.  This would motivate the merging of small cells following \textcite{yamazaki-satomura2010}.

\begin{itemize}
\item true cut cell mesh so that we can compare with others
\item explicit advection rather than implicit (so ExnerFoam and scalarTransport are comparable)
\item mathematically investigate effects of metric term, try to show they're equivalent
\item interpolate velocity field from TF GW test and prescribe that flow for a wobbly advection test
\item resting atmosphere: errors on regular grid should be lower
\item resting atmosphere: energy oscillations and increases
\item 4km high mountains for resting atmosphere test.
\item Charney-Phillips staggering
\item three dimensions (what kind of tests?  \textcite{lock2012} did some 3D tests that we could compare against.  I read a bit about 3D gravity waves in Topographic effects in stratified flows, Baines 1995) 
\item is it possible to include metric terms in the FV OpenFoam model?  then we could see what the effect of coord transforms is
\end{itemize}
